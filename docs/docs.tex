% Preamble
\documentclass[11pt]{PyRollDocs}
\usepackage{pythontex}

\addbibresource{refs.bib}
% Document
\begin{document}

    \title{The Byon Lee Wear Model PyRolL Plugin}
    \author{Christoph Renzing}
    \date{\today}

    \maketitle

    The \emph{Byon Lee Wear Model} Plugin implements a wear model for the oval - round - oval groove series.
    The model was published by Byon et al.~\cite{Byon2007, Byon2008, Byon2008a} a focuses solely on the mechanical and geometrical factors are considered.
    Metallurgical influences like corrosion or thermal fatigue are neglected.
    The main parameters that are considered are:

    \begin{description}
        \item[$F_{R}$] Roll Force
        \item[$L_{d}$] Contact Length
        \item[$H_{s}$] Shore Hardness
        \item[$N_{b}$] Number of rolled billets
    \end{description}

    The key difference between other models like the Wear model of e.g. Archard, is that the proposed model also allows to calculate the resulting wear contour.
    Byon et al. therefore assumes, that for round grooves, the wear contour takes a parablic form.
    The wear radius for oval grooves is calculated using the following \autoref{eq:1}:

    \begin{equation}
        R_{w,g} = R_2 \cdot J_w + R_{p, 0} \cdot \left( 1 - J_w \right)
        \label{eq:1}
    \end{equation}

    In this equation, $R{w,g}$ is the worn radius of the groove profile.
    $R_2$ is the second radius of the oval groove, $J_w$ is a weight function introduced by Byon et al..
    $R_{p, 0}$ is the radius of the incoming round profile.

    The weight function $J_w$ for oval grooves is calculated as:

    \begin{equation}
        J_w = 1 - \kappa \cdot \left( \frac{F_R^2 \cdot L_d \cdot N_b}{H_s} \right)
        \label{eq:2}
    \end{equation}

    The derivation of the equation can be found in the author's original publication~\cite{Byon2008a}.
    Further, the authors provide a formula to calculate the offset of the wear profile from the initial groove contour.
    The implementation of this part was neglected due to usage of the \emph{shapely} python package witch allows for direct fitting of the contour to the desired points.
    These points are the so-called detachment or separation points, witch mark the point were the profile detaches from the original groove.\\
    As for round grooves, the used equations are:

    \begin{equation}
        R_{w,g} = R_2 \cdot J_w + 0.75 \cdot R_{b} \cdot \left( 1 - J_w \right)
        \label{eq:3}
    \end{equation}

    In this equation $R_b$ is the bulge radius of the incoming oval profile.
    This bulge radius can be calculated can be calculated according to the model of \textcite{Schmidt2017}, \textcite{Byon2017} or \textcite{Lee2001a}.
    To calculate the weight \autoref{eq:2} is used.
    The correction coefficients $\kappa$ were measured by the autor from various rolling trials.
    For round - oval passes the value is $\kappa = 19.6e-12$ and for oval - round passes $\kappa =  35.9e-12$.
    Further, for round - oval roll passes the value $0.75$ is a empirical coefficient as stated by the original author.


    \section{Usage instructions}\label{sec:usage-instructions}

    The plugin can be loaded under the name \py/pyroll.byon-lee-wear-model/.
    The functionality of the plugin should need the following values to operate: \py/RollPass.Roll.shore_hardness/ for the shore hardness of the roll material and \py/RollPass.rolled_billets/ a integer number of rolled billets.



    \printbibliography

\end{document}